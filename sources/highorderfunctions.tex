\usepackage{etex}
%\documentclass{article}
%\usepackage{beamerarticle}
%\usepackage{pstricks,pst-node} % PSTricks package
%\usepackage[turkish]{babel}
\usepackage[utf8]{inputenc}
\usepackage{listings}
%\includeonlyframes{current}
\usepackage[graph,arrow,curve]{xy}

\newcommand{\aequiv}{\mathop{\equiv_{\alpha}}}

\mode<article>
{
  \usepackage{fullpage}
  \usepackage{pgf}
  \usepackage{hyperref}
}

\mode<presentation>
{
  \usetheme{metuceng}
  \setbeamercovered{transparent}
}


\titlegraphic{\insertmetutitle\insertlicense}


\title{Programming Language Concepts}
\author{Onur Tolga Şehitoğlu}
\institute[METU]{Computer Engineering}
\subtitle{Higher Order Functions}
\date{}

\begin{document}
\lstset{language=C,
        basicstyle=\scriptsize\ttfamily,
        keywordstyle=\color{blue!50!black}\bfseries,
        identifierstyle=\color{blue!60!green}\sffamily,
        stringstyle=\color{red!70!green}\ttfamily,
	commentstyle=\color{blue!30!white}\itshape,
        showstringspaces=true}
\setbeamercolor{hexample}{bg=green!5!white,fg=black}%
\setbeamercolor{cexample}{bg=blue!5!white,fg=black}%
\setbeamercolor{pexample}{bg=orange!5!white,fg=black}%
\setbeamercolor{oexample}{bg=violet!5!white,fg=black}%

 \frame[plain]{\maketitle}

 \begin{frame}
 \frametitle{Outline}
\small
 \tableofcontents
 \end{frame}

\section{Lambda Calculus}
\begin{frame}
\frametitle{Lambda Calculus}
\begin{itemize}
 \item 1930's by Alonso Church and Stephen Cole Kleene
 \item Mathematical foundation for computatibility and recursion
 \item Simplest functional paradigm language
 \item $\lambda var . expr$ \\
	defines an anonymous function. Also called \structure{lambda 
	abstraction}
 \item $expr$ can be any expression with other lambda abstractions and 
	applications. Applications are one at a time.
 \item $(\lambda x . \lambda y . x+y) \; 3 \; 4$ 
\end{itemize}
\end{frame}

\begin{frame}
 \begin{itemize}
  \item In `$\lambda var . expr$ ' all free occurences of $var$ is bound by the $\lambda var$.
  \item Free variables of expression $FV(expr)$\\
	\begin{itemize}
	\item $FV(name) = \{ name \} $ if $name$ is a variable
        \item $FV(\lambda name. expr)= FV(expr) - \{name\}$
        \item $FV(M \; N) = FV(M) \cup FV(N)$
        \end{itemize}
  \item \structure{$\alpha$ conversion}: expressions with all bound names changed to another name are equivalent:\\
	$\lambda f. f \; x \aequiv \lambda y. y \; x \aequiv \lambda z. z \; x$ \\
        $\lambda x. x + (\lambda x. x+y) \aequiv \lambda t. t + (\lambda x. x+y)
	\aequiv \lambda t. t + (\lambda u. u+y)$\\
        $\lambda x. x + (\lambda x. x+y) \not\equiv_{\alpha} \lambda x. x + (\lambda x. x+t)$
 \end{itemize}
\end{frame}

\begin{frame}
 \frametitle{ $\beta$ Reduction}
\begin{itemize}
 \item Basic computation step, function application in $\lambda$-calculus
 \item Based on substitution. All bound occurences of $\lambda$ variable parameter is substituted by the
       actual parameter
 \item $(\lambda x . M) N \mapsto_{\beta} M [x/N]$ (all $x$'s once bound by lambda are substituted with $N$).
 \item $(\lambda x.(\lambda y.y+(\lambda x.x+1)\;y) (x+2))\; 5$
 \item If no further $\beta$ reduction is possible, it is called a normal form.
 \item There can be different reduction strategies but should reduce to 
	same normal form. (Church Rosser property)
\end{itemize}
\end{frame}

\begin{frame}
All possible reductions of a $\lambda$-expression. All reduce to the
same normal form.\\
\tiny
\xygraph{[]!{<14mm,0mm>:<0mm,8mm>::}
{\lambda x.(\lambda y.y+(\lambda x.x+1~y)~~(x+2))~~5}
 ( :@{|->}[dll] {\lambda y.y+(\lambda x.x+1~~y)~~(5+2)} 
	( :@{|->}[dl] {5+2+(\lambda x.x+1~~(5+2))}
		( :@{|->}[ddddrrr] {5+2+5+2+1}="bot" ) ,
	  :@{|->}@(dr,ur)[dddl] {\lambda y.y+y+1~~(5+2)} :@{|->}"bot"
	),
   :@{|->}[dd]  {\lambda x.x+2+(\lambda x.x+1~~(x+2))~5} 
	(
   	:@{|->}[dl]  {5+2+(\lambda x.x+1~~(5+2))} :@{|->}"bot",
   	:@{|->}[dr]  {\lambda x.x+2+x+2+1~~5} :@{|->}"bot"
	),
   :@{|->}[drr] {\lambda x.(\lambda y.y+y+1~~(x+2))~~5} (
	   :@{|->}[ddr] {\lambda y.y+y+1~~(5+2)} :@{|->}@(dr,r)"bot",
	   :@{|->}[ddd] {\lambda x.x+2+x+2+1~~5} :@{|->}"bot")
 )
}
\end{frame}



\defverbatim[colored]\codecomposeH{
\begin{lstlisting}[language={Haskell}]
f x = x+x
g x = x*x
compose func1 func2 x = func1 (func2 x)
t = compose f g
u = compose g f
\end{lstlisting}}
\section{Introduction}
\begin{frame}
\frametitle{Introduction}
\begin{itemize}[<+->]
\item Mathematics:\\
	$(f \circ g )(x) = f(g(x))$ , $(g \circ f) (x) = g(f(x))$
\item ``$\circ$'' : Gets two unary functions and composes a new function.
A function getting two functions and returning a new function.
\item in Haskell:
\begin{beamercolorbox}{hexample}
\codecomposeH
\end{beamercolorbox}
\item \texttt{\small t 3 = (3*3)+(3*3) = 18}\\
	\texttt{\small u 3 = (3+3)*(3+3) = 36}
\item \texttt{compose}: $(\beta \rightarrow \gamma) \rightarrow (\alpha \rightarrow \beta) \rightarrow \alpha \rightarrow \gamma$
\end{itemize}
\end{frame}

\begin{frame}
\begin{itemize}[<+->]
 \item ``\texttt{compose}'' function is a function getting two functions as
 parameters and returning a new function.
 \item Functions getting one or more functions as parameters are called
 \structure{Higher Order Functions}.
 \item Many operations on functional languages are repetition of a basic
 task on data structures.
 \item Functions are first order values $\rightarrow$ new general purpose 
 functions that uses other functions are possible.
\end{itemize}
\end{frame}

\defverbatim[colored]\codecurryH{
\begin{lstlisting}[language={Haskell}]
curry f x y = f(x,y)
add (x,y) = x+y
increment = curry add 1
---
increment 5
6
\end{lstlisting}}
\section{Functions}
\subsection{Curry}
\begin{frame}
\frametitle{Functions/Curry}
\begin{itemize}
\item Cartesian form vs curried form:\\
	$\alpha \times \beta \rightarrow \gamma$ vs 
	$\alpha \rightarrow \beta \rightarrow \gamma$
\item Curry function gets a binary function in cartesian form and converts it to curried form.
\begin{beamercolorbox}{hexample}
\codecurryH
\end{beamercolorbox}
 \item \texttt{curry}:
	$(\alpha \times \beta \rightarrow \gamma) \rightarrow
	\alpha \rightarrow \beta \rightarrow \gamma$
 \item Haskell library includes it as \structure{\texttt{curry}}.
\end{itemize}
\end{frame}

\defverbatim[colored]\codemapH{
\begin{lstlisting}[language={Haskell}]
square x = x*x
day no = case no of 1 -> "mon" ; 2 -> "tue" ; 3 -> "wed"; 
        4 -> "thu" ; 5 -> "fri" ; 6 -> "sat" ; 7 -> "sun"
map func [] = []
map func (el:rest) = (func el):(map func rest)
----
map square [1,3,4,6]
[1,9,6,36]
map day [1,3,4,6]
["mon","wed","thu","sat"]
\end{lstlisting}}

\subsection{Map}
\begin{frame}
\frametitle{Functions/Map}
\begin{beamercolorbox}{hexample}
\codemapH
\end{beamercolorbox}
\begin{itemize}
 \item \texttt{map}:$(\alpha \rightarrow \beta) \rightarrow [\alpha] \rightarrow  [\beta]$
 \item Gets a function and a list. Applies the function to all elements and 
 returns a new list of results.
 \item Haskell library includes it as \structure{\texttt{map}}.
\end{itemize}
\end{frame}

\defverbatim[colored]\codefilterH{
\begin{lstlisting}[language={Haskell}]
iseven x = if mod x 2 == 0 then True else False 
isgreater x = x>5

filter func [] = []
filter func (el:rest) = if func el then  
                                el:(filter func rest)
                          else  (filter func rest)
----
filter iseven [1,2,3,4,5,6,7]
[2,4,6]
filter isgreater [1,2,3,4,5,6,7]
[6,7]
\end{lstlisting}}

\subsection{Filter}
\begin{frame}
\frametitle{Functions/Filter}
\begin{beamercolorbox}{hexample}
\codefilterH
\end{beamercolorbox}
\begin{itemize}
 \item \texttt{filter}:$(\alpha \rightarrow Bool) \rightarrow [\alpha] \rightarrow  [\alpha]$
 \item Gets a boolean function and a list. Returns a list with only members
 evaluated to \texttt{True} by the boolean function.
 \item Haskell library includes it as \structure{\texttt{filter}}.

\end{itemize}
\end{frame}

\defverbatim[colored]\codereduceH{
\begin{lstlisting}[language={Haskell}]
sum x y = x+y
product x y = x*y

reduce func s [] = s
reduce func s (el:rest) = func el (reduce func s rest)
----
reduce sum 0 [1,2,3,4]
10                             // 1+2+3+4+0
reduce product 1 [1,2,3,4]
24                             // 1*2*3*4*1
\end{lstlisting}}

\subsection{Reduce}
\begin{frame}
\frametitle{Functions/Reduce (Fold Right)}
\begin{beamercolorbox}{hexample}
\codereduceH
\end{beamercolorbox}
\begin{itemize}
 \item \texttt{reduce}:$(\alpha \rightarrow \beta \rightarrow \beta) \rightarrow \beta  \rightarrow  [\alpha] \rightarrow \beta$
 \item Gets a binary function, a list and a seed element. Applies function
 to all elements right to left with a single value.\\
	$reduce\;f\;s\;[a_1,a_2,...,a_n] = f\;a_1\;(f\;a_2\;(....\;(f\;a_n\;s)))$
 \item Haskell library includes it as  \structure{\texttt{foldr}}.
\end{itemize}
\end{frame}


\begin{frame}
 \begin{itemize}
  \item Sum of a numbers in a list:\\
\texttt{listsum = reduce sum 0}
  \item Product of a numbers in a list:\\
\texttt{listproduct = reduce product 1}
  \item Sum of squares of a list:\\
\texttt{squaresum x = reduce sum 0 (map square x)}
 \end{itemize}
\end{frame}


\defverbatim[colored]\codefoldlH{
\begin{lstlisting}[language={Haskell}]
subtract x y = x - y
foldl func s [] = s
foldl func s (el:rest) = 
           foldl func (func s el) rest
----
reduce subtract 0 [1,2,3,4]
-2                             // 1-(2-(3-(4-0)))
foldl subtract 0 [1,2,3,4]
-10                            // ((((0-1)-2)-3)-4)
\end{lstlisting}}
\subsection{Fold Left}
\begin{frame}
\frametitle{Functions/Fold Left}
\begin{beamercolorbox}{hexample}
\codefoldlH
\end{beamercolorbox}
\begin{itemize}
 \item \texttt{foldl}:$(\alpha \rightarrow \beta \rightarrow \alpha) \rightarrow \alpha  \rightarrow  [\beta] \rightarrow \alpha$
 \item Reduce operation, left associative.: \\
	$reduce\;f\;s\;[a_1,a_2,...,a_n] = f\;(f\;(f\;...(f\;s\;a_1)\;a_2\;...))\;a_n$
 \item Haskell library includes it as \structure{\texttt{foldl}}.
\end{itemize}
\end{frame}


\defverbatim[colored]\codeiterH{
\begin{lstlisting}[language={Haskell}]
twice x = 2*x

iterate func s 0 = s
iterate func s n = func (iterate func s (n-1))
----
iterate twice 1 4
16                       // twice (twice ( twice (twice 1))
iterate square 3 3
6561                     // square (square (square 3))
\end{lstlisting}}

\subsection{Iterate}
\begin{frame}
\frametitle{Functions/Iterate}
\begin{beamercolorbox}{hexample}
\codeiterH
\end{beamercolorbox}
\begin{itemize}
 \item \texttt{iterate}:$(\alpha \rightarrow \alpha) \rightarrow \alpha \rightarrow int  \rightarrow  \alpha$
 \item Applies same function for given number of times, starting with the
 initial seed value.
	$iterate\;f\;s\;n = f^n\;s = \underbrace{f\;(f\;(f\;...(f\;s))}_n$
\end{itemize}
\end{frame}


\defverbatim[colored]\codeforH{
\begin{lstlisting}[language={Haskell}]
for func s m n = 
      if m>n then s
      else for func (func s m) (m+1) n

----
for sum 0 1 4
10        // sum (sum (sum (sum 0 1) 2) 3) 4
for product 1 1 4 
24        // product (product (product (product 1 1) 2) 3) 4
\end{lstlisting}}
\subsection{Value Iteration (for)}
\begin{frame}
\frametitle{Functions/Value Iteration (for)}
\begin{beamercolorbox}{hexample}
\codeforH
\end{beamercolorbox}
\begin{itemize}
 \item \texttt{for}:$(\alpha \rightarrow int \rightarrow \alpha) \rightarrow \alpha \rightarrow int  \rightarrow  int \rightarrow \alpha$
 \item Applies a binary integer function to a range of integers in order.\\
 $for\;f\;s\;m\;n = f(f\;(f\;(f\;(f\;s\;m)\;(m+1))\;(m+2))\;...)\;n$
\end{itemize}
\end{frame}

\begin{frame}
 \begin{itemize}
  \item multiply (with summation):\\
\texttt{multiply x = iterate (sum x) x}
  \item integer power operation (Haskell '\lstinline{^}'):\\
\texttt{power x = iterate (product x) x}
  \item sum of values in range 1 to  n:\\
\texttt{seriessum = for sum 0 1}
  \item Factorial operation:\\
\texttt{factorial = for product 1 1}
 \end{itemize}
\end{frame}


\defverbatim[colored]\codesortC{
\begin{lstlisting}[language={C}]
typedef struct Person { char name[30]; int no;} person;
int cmpnmbs(void *a, void *b) {
    person *ka=(person *)a; person *kb=(person *)b;
    return ka->no - kb->no;
}
int cmpnames(void *a, void *b) {
    person *ka=(person *)a; person *kb=(person *)b;
    return strncmp(ka->name,kb->name,30);
}
int main() {       int i;
    person list[]={{"veli",4},{"ali",12},{"ayse",8},
                  {"osman",6},{"fatma",1},{"mehmet",3}};
    qsort(list,6,sizeof(person),cmpnmbs);
    ...
    qsort(list,6,sizeof(person),cmpnames);
    ...
}
\end{lstlisting}}
\section{Higher Order Functions in C}
\begin{frame}
\frametitle{Higher Order Functions in C}
C allows similar definitions based on function pointers.
Example: \structure{bsearch()} and \structure{qsort()} funtions in C library.
\begin{beamercolorbox}{cexample}
\codesortC
\end{beamercolorbox}
\end{frame}

\defverbatim[colored]\codefibH{
\begin{lstlisting}[language={Haskell}]
fib n = let f (x,y) = (y,x+y)
            (a,b) = iterate f (0,1) n
        in b
----
fib 5      // f(f(f(f(0,1)))) 
8          //(0,1)->(1,1)->(1,2)->(2,3)->(3,5)->(5,8)
\end{lstlisting}}
\section{Some examples}
\subsection{Fibonacci}
\begin{frame}
\frametitle{Fibonacci}
 Fibonacci series: 1 1 2 3 5 8 13 21 .. \\
 $fib(0) = 1 \;\;; \; fib(1) = 1\;\;; \; fib(n) = fib(n-1)+fib(n-2)$\\
\begin{beamercolorbox}{hexample}
\codefibH
\end{beamercolorbox}
\end{frame}

\defverbatim[colored]\codesortH{
\begin{lstlisting}[language={Haskell}]
notfunc f x y = not (f x y)

sort _ [] = []
sort func (x:xs) = (sort func (filter (func x) xs)) ++ 
                (x: (sort func (filter ((notfunc func) x) xs)))
---
sort (>) [5,3,7,8,9,3,2,6,1]
[1,2,3,3,5,6,7,8,9]
sort (<) [5,3,7,8,9,3,2,6,1]
[9,8,7,6,5,3,3,2,1]
\end{lstlisting}}

\subsection{Sorting}
\begin{frame}
\frametitle{Sorting}
 Quicksort:
\begin{enumerate}
 \item First element of the list is \texttt{x} and rest is \texttt{xs}
 \item select smaller elements of \texttt{xs} from \texttt{x}, sort them and put before
 \texttt{x}.
 \item select greater elements of \texttt{xs} from \texttt{x}, sort them and put after
 \texttt{x}.
\end{enumerate}
\begin{beamercolorbox}{hexample}
\codesortH
\end{beamercolorbox}
\end{frame}

\defverbatim[colored]\codetersH{
\begin{lstlisting}[language={Haskell}]
reverse1 [] = []
reverse1 (x:xs) = (reverse1 xs) ++ [x]

reverse2 x = reverse2' x []  where
        reverse2' [] x = x
	reverse2' (x:xs) y = reverse2' xs (x:y)
---
reverse1 [1..10000]        // slow
reverse2 [1..10000]        // fast
\end{lstlisting}}

\subsection{List Reverse}
\begin{frame}
\frametitle{List Reverse}{\scriptsize
\begin{itemize}[<+->]
\item  Taking the reverse
\begin{itemize}\scriptsize
 \item First element is \texttt{x} rest is \texttt{xs}
 \item Reverse the \texttt{xs},  append \texttt{x} at the end
\end{itemize}
Loose time for appending \texttt{x} at the end at each step ($N$ times append of size $~ N$).
\item Fast version, extra parameter (initially empty list) added:
\begin{itemize}\scriptsize
 \item Take the first element, insert at the beginning of the extra parameter.
 \item Recurse rest of the list with the new extra parameter. 
 \item When recursion at the deepest, return the extra parameter.
\end{itemize}
Inserts to the beginning of the list at each step. Faster ($N$ times insertion)
\end{itemize}}
\begin{beamercolorbox}{hexample}
\codetersH
\end{beamercolorbox}
\end{frame}
\end{document}
