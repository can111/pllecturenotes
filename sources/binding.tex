%\documentclass[compress,dvips,xcolor=table]{beamer}
\usepackage{etex}
%\documentclass{article}
%\usepackage{beamerarticle}
%\usepackage{pstricks,pst-node} % PSTricks package
%\usepackage[turkish]{babel}
\usepackage[utf8]{inputenc}
\usepackage{listings}
\usepackage{multicol}
%\includeonlyframes{current}

\def\circtxt#1{$\mathalpha \bigcirc \mkern-13mu \mathtt #1$}

\mode<article>
{
  \usepackage{fullpage}
  \usepackage{pgf}
  \usepackage{hyperref}
}

\mode<presentation>
{
  \usetheme{metuceng}

%  \setbeamercovered{transparent}
}


\title{Programming Language Concepts/Binding and Scope}
\author{Onur Tolga Şehitoğlu}
\institute[ODTÜ]{Bilgisayar Mühendisliği}
\subject{Binding and Scope}
\date{}

\begin{document}
\lstset{language=C,
        basicstyle=\scriptsize\ttfamily,
        keywordstyle=\color{blue!50!black}\bfseries,
        identifierstyle=\color{blue!60!green}\sffamily,
        stringstyle=\color{red!70!green}\ttfamily,
	commentstyle=\color{blue!30!white}\itshape,
        showstringspaces=true}
\setbeamercolor{hexample}{bg=green!5!white,fg=black}%
\setbeamercolor{cexample}{bg=blue!5!white,fg=black}%
\setbeamercolor{pexample}{bg=orange!5!white,fg=black}%
\setbeamercolor{oexample}{bg=violet!5!white,fg=black}%

 \frame[plain]{\maketitle}
 \begin{frame}
 \frametitle{Outline}
 \begin{multicols}{2}
 \tableofcontents
 \end{multicols}
 \end{frame}
\section{Binding}
\begin{frame}
\frametitle{Binding}
\begin{itemize}[<+->]
\item Most important feature of high level languages: programmers able to
give names to program entities (variable, constant, function, type, ...).
These names are called \structure{identifiers}.
\item definition of an identifier $\leftrightarrows$ used position of an
identifier. Formally:
binding occurrence $\leftrightarrows$ applied occurrence.
\item Identifiers are declared once, used $n$ times.
\item Language should map which corresponds to which.
\item \structure{Binding:} Finding the corresponding binding occurrence
(definition/declaration) for an applied occurrence (usage) of an identifier.
\end{itemize}
\end{frame}

\defverbatim[colored]\codescopefailC{
\begin{lstlisting}[language={C},escapechar=\#]
double f,y;
int f() {  #\alert{$\times$ error!}#
   ...
}
double y; #\alert{$\times$ error!}#
\end{lstlisting}}
\defverbatim[colored]\codescopeokC{
\begin{lstlisting}[language={C},escapechar=\#]
double y;
int f() { 
   double f;   #\color{green!60!black}{$\surd$ OK}#
   int y ;     #\color{green!60!black}{$\surd$ OK.}#
}
\end{lstlisting}}

\begin{frame}
 for binding:
\begin{enumerate}[<+->]
 \item \structure{Scope of identifiers} should be known. What is the block
 structure? Which blocks the identifier is available.
 \item  What will happen if we use same identifier name again\\
 ``C forbids reuse of same identifier name in the same scope. Same name can
 be used in different nested blocks. The identifier inside
 \structure{hides} the outside identifier''.
\begin{columns}
 \begin{column}{.5\linewidth}
  \begin{beamercolorbox}{cexample}
     \codescopefailC
  \end{beamercolorbox}
 \end{column}
\begin{column}{.5\linewidth}
  \begin{beamercolorbox}{cexample}
 \codescopeokC
\end{beamercolorbox}
\end{column}
\end{columns}
\end{enumerate}
\end{frame}

\defverbatim[colored]\codeenvC{
\begin{lstlisting}[language={C},escapechar=\#]
struct Person { ... } x;
int f(int a) { 
   double y;
   int x;
   ... #\circtxt{1}#
}
int main() {
   double a;
   ... #\circtxt{2}#
}
\end{lstlisting}}
\section{Environment}
\begin{frame}
 \frametitle{Environment}
\begin{itemize}[<+->]
 \item \structure{Environment}: The set of binding occurrences that are accessible at a
 point in the program.
\item Example: \\
\begin{columns}

\begin{column}{.45\linewidth}
\begin{beamercolorbox}{cexample}
 \codeenvC
\end{beamercolorbox}
\end{column}

\begin{column}{.55\linewidth}\small
\noindent
O(\circtxt{1})=\{struct Person $\mapsto$ type,\\
 x $\mapsto$ int, f $\mapsto$ func, a $\mapsto$ int, \\
y $\mapsto$ double\}\\[2em]
O(\circtxt{2})=\{struct Person $\mapsto$ type,\\
x $\mapsto$ struct Person, f $\mapsto$ func, \\
a $\mapsto$ double, main $\mapsto$ func\}\\
\end{column}
\end{columns}

\end{itemize}
\end{frame}

\section{Block Structure}
\begin{frame}
 \frametitle{Block Structure}
\begin{itemize}[<+->]
 \item  Program blocks define the scope of the identifiers declared inside. (boundary of the
 definition validity) For variables, they also define the lifetime. 
 \item Languages may have different block structures:
\begin{itemize}
\item[\bf C] function definitions and command blocks (\{ ... \}) define local scopes. Also each
source code define a block.
\item[\bf Java] Class definitions, class member function definitions, block commands define
local scopes. Nested function definitions and namespaces possible.
\item[\bf Haskell] `\texttt{let {\em definitions\/} in {\em expression\/}}' defines a block
expression. Also `\texttt{ {\em expression \/} where {\em definitions\/}}' defines a block
expression. (the definitions have a local scope and not accessible outside of the expression)
\end{itemize}
\item Block structure of the language is defined by the organization of the blocks.
\end{itemize}
\end{frame}

\subsection{Monolithic block structure}
\begin{frame}
 \frametitle{Monolithic block structure}
\begin{itemize}
 \item Whole program is a block. All identifiers have global scope starting from the
 definition.
 \item \structure{Cobol} is a monolithic block structure language.\\
\texttt{\begin{tabular}{|>{\columncolor{blue!15!white}}p{\linewidth}|}\hline
   int x;\\
   int y;\\
   ....\\
   ....\\ \hline
\end{tabular}}
\item In a long program with many identifiers, they share the same scope and they need to be
distinct.
\end{itemize}

\end{frame}

\subsection{Flat block structure}
\begin{frame}
 \frametitle{Flat block structure}
\begin{itemize}
 \item Program contains the global scope and only a single level local scope of function
 definitions. No further nesting is possible.
 \item \structure{Fortran} and partially \structure{C} has flat block structure.\\
{\tiny \tt
\begin{tabular}{|>{\columncolor{blue!15!white}}p{\linewidth}|} \hline
   int x;\\
   int y;\\
   int f() \\
   \ \begin{tabular}{|>{\columncolor{blue!10!white}}p{.8\linewidth}|} \hline
   \{\ int a;\\
   \ \ double b;\\
   ... \\ 
   \} \\ \hline
   \end{tabular} \\[1em]
   int g() \\
   \ \begin{tabular}{|>{\columncolor{blue!10!white}}p{.8\linewidth}|} \hline
   \{ \ int a;\\
   \ \ double b;\\
   ... \\ 
   \}\\ \hline
   \end{tabular}\\
   ....\\ \hline
\end{tabular}}
\end{itemize}
\end{frame}

\subsection{Nested block structure}
\begin{frame}
 \frametitle{Nested block structure}
\begin{itemize}
 \item Multiple blocks with nested local scopes can be defined.
 \item \structure{Pascal} and \structure{Java} have nested block structure.\\
{\tiny \tt
\begin{tabular}{|>{\columncolor{blue!15!white}}p{\linewidth}|} \hline
   int x;\\
   int f() \\
   \ \begin{tabular}{|>{\columncolor{blue!10!white}}p{.8\linewidth}|} \hline
   \{\ int a;\\
     double g() \\
\ \begin{tabular}{|>{\columncolor{blue!5!white}}p{.8\linewidth}|} \hline
   \{\ int x;\\
   ... \\ 
   \} \\ \hline
   \end{tabular} \\[1em]
   ... \\ 
   \} \\ \hline
   \end{tabular} \\[1em]
   int g() \\
   \ \begin{tabular}{|>{\columncolor{blue!10!white}}p{.8\linewidth}|} \hline
   \{ \ int h()\\
   \begin{tabular}{|>{\columncolor{blue!5!white}}p{.8\linewidth}|} \hline
   \{\ int x;\\
   ... \\ 
   \} \\ \hline
   \end{tabular} \\[1em]
   ... \\ 
   \}\\ \hline
   \end{tabular}\\
   ....\\ \hline
\end{tabular}}
\item C block commands can be nested.
\item GCC extensions (\structure{Not C99 standard!}) to C allow nested function definitions.
\end{itemize}
\end{frame}


\defverbatim[colored]\codehidingC{
\begin{lstlisting}[language={C},escapechar=\#]
int x,y;
int f(double x) {
   ...             // parameter x hides global x in f()
}
int g(double a) {
    int y;         // local y hides global y in g()
    double f;      // local f hides global f() in g()
    ...
}
int main() {
    int y;        // local y hides global y in main()
}
\end{lstlisting}}
\section{Hiding}
\begin{frame}
 \frametitle{Hiding}
\begin{itemize}
 \item Identifiers defined in the inner local block hides the outer block identifiers with the
 same name during their scope. They cannot be accessed within the inner block.
\begin{beamercolorbox}{cexample}
\codehidingC
\end{beamercolorbox}
\end{itemize}
\end{frame}

\section{Static vs Dynamic Scope/Binding}
\begin{frame}
 \frametitle{Static vs Dynamic Scope/Binding}
The binding and scope resolution is done at compile time or run time? Two options:
\begin{enumerate}
 \item Static binding, static scope
 \item Dynamic binding, dynamic scope
\end{enumerate}
\begin{itemize}
 \item First defines scope and binding based on the lexical structure of the program and
 binding is done at compile time.
 \item Second activates the definitions in a block during the execution of the block. The
 environment changes dynamically at run time as functions are called and returned.
\end{itemize}
\end{frame}

\defverbatim[colored]\codestaticbindC{
\begin{lstlisting}[language={C},escapechar=\#]
int x=1,y=2;
int f(int y) {
     y=x+y;      
     return x+y;
}
int g(int a) {
    int x=3;       
    y=x+x+a;    x=x+y;    y=f(x);
    return x;
}
int main() {
    int y=0;    int a=10;  
    x=a+y;    y=x+a;    a=f(a);    a=g(a);
    return 0;
}
\end{lstlisting}}
\defverbatim[colored]\codestaticbindCR{
\def\X{\color{green!60!black}\bf x}
\def\gX{\color{red!60!black}\bf x}
\def\Y{\color{green!60!black}\bf y}
\def\mY{\color{red!60!black}\bf y}
\def\fY{\color{yellow!60!black}\bf y}
\def\gA{\color{yellow!60!black}\bf a}
\def\mA{\color{red!60!black}\bf a}
\begin{lstlisting}[language={C},escapechar=\#]
int #\X#=1,#\Y#=2;
int f(int #\fY#) {
     #\fY#=#\X#+#\fY#;                 /* #\X# global, #\fY# local */
     return #\X#+#\fY#;
}
int g(int #\gA#) {
    int #\gX#=3;                 /* #\gX# local, #\Y# global */
    #\Y#=#\gX#+#\gX#+#\gA#;    #\gX#=#\gX#+#\Y#;    #\Y#=f(#\gX#);
    return #\gX#;
}
int main() {
    int #\mY#=0;    int #\mA#=10;        /* #\X# global #\mY# local */
    #\X#=#\mA#+#\mY#;    #\mY#=#\X#+#\mA#;    #\mA#=f(#\mA#);    #\mA#=g(#\mA#);
    return 0;
}
\end{lstlisting}}
\subsection{Static binding}
\begin{frame}
 \frametitle{Static binding}
 \begin{itemize}
  \item Programs shape is significant. Environment is based on the position in the source
  (lexical scope)
  \item Most languages apply static binding (C, Haskell, Pascal, Java, ...)
   \only<2>{\begin{beamercolorbox}{cexample}
   \codestaticbindCR
   \end{beamercolorbox}}
   \only<1>{\begin{beamercolorbox}{cexample}
   \codestaticbindC
   \end{beamercolorbox}}
 \end{itemize}
\end{frame}

\defverbatim[colored]\codedynamicbindC{
\begin{lstlisting}[language={C},escapechar=\#,basicstyle=\tiny\ttfamily,
		numbers=left,numberstyle=\tiny,stepnumber=1]
int x=1,y=2;
int f(int y) {
     y=x+y;      
     return x+y;
}
int g(int a) {
    int x=3;       
    y=x+x+a;  x=x+y;    
    y=f(x);
    return x;
}
int main() {
    int y=0;  int a=10;  
    x=a+y;    y=x+a;    
    a=f(a);   a=g(a);
    return 0;
}
\end{lstlisting}}
\subsection{Dynamic binding}
\begin{frame}
 \frametitle{Dynamic binding}
 \begin{itemize}
  \item Functions called update their declarations on the environment at \structure{run-time}.
Delete them on return. Current stack of activated
blocks is significant in binding.
  \item Lisp and some script languages apply dynamic binding. 
  \begin{columns}
  \begin{column}{.35\linewidth}
   \begin{beamercolorbox}{cexample}
   \codedynamicbindC
   \end{beamercolorbox}
   \end{column}
   \begin{column}{.65\linewidth}
   \only<2->{
   \def\T{\rule{0pt}{1em}\hspace*{1em}}
   \noindent\tiny\begin{tabular}{rll}
    & Trace & Environment (without functions)\\ \hline
    & initial & \{x:GL, y:GL \} \\ \rowcolor{blue!5}
   12& call main & \{x:GL, y:main, a:main \} \\ \rowcolor{blue!15}
   15&\T call f(10)  & \{x:GL, y:f , a:main \} \\ \rowcolor{blue!15}
   4&\T return f : 30 & back to environment before f  \\ \rowcolor{blue!5}
   15& in main & \{x:GL, y:main, a:main \} \\ \rowcolor{blue!10}
   15&\T call g(30) & \{x:g, y:main, a:g  \} \\ \rowcolor{blue!25}
   9&\T\T call f(39) & \{x:g, y:f, a:g  \} \\ \rowcolor{blue!25}
   4&\T\T return f : 117 & back to environment before f\\ \rowcolor{blue!10}
   9&\T in g  & \{x:g, y:main, a:g  \} \\ \rowcolor{blue!10}
   10&\T return g : 39 & back to environment before g \\ \rowcolor{blue!5}
   15& in main & \{x:GL, y:main, a:main\} \\ \rowcolor{blue!5}
   16& return main & x:GL=10, y:GL=2, y:main=117, a:main=39 \\
   \end{tabular}
   }
   \end{column}
   \end{columns}
 \end{itemize}
\end{frame}

\begin{frame}[fragile]
\begin{itemize}
\item It gets more complicated if nested functions are allowed:
\begin{beamercolorbox}{hexample}
\begin{lstlisting}[language=Haskell]
let x = 5
    y = 10
    func x = x * x + other y
    other x = x + x
in func x + let other x = x - 1
                y = 2
            in func x
\end{lstlisting}
\end{beamercolorbox}
\end{itemize}
\end{frame}

\section{Binding Process}
\begin{frame}
\frametitle{Binding Process}
\begin{itemize}
\item Language processor keeps track of current environment in a data structure called \structure{Symbol Table} or \structure{Identifier Table}
\item Symbol table maps identifier strings to their type and binding.
\item Each new block introduces its declarations/bindings to the symbol table and on exit, they are cleared.
\item Usually implemented as a Hash Table. 
\item For static binding, Symbol Table is a compile time data structure and maintained during different stages of compilation.
\item For dynamic binding, symbol table is maintained at run time.
\end{itemize}
\end{frame}

\section{Declarations}
\begin{frame}
\frametitle{Declarations}
\begin{itemize}
 \item Definitions vs Declarations
 \item Sequential declarations
 \item Collateral declarations
 \item Recursive declarations
 \item Collateral recursive declarations
 \item Block commands
 \item Block expressions
\end{itemize}
\end{frame}

\subsection{Definitions and Declarations}
\begin{frame}
\frametitle{Definitions and Declarations}
\begin{itemize}
 \item \structure{Definition:} Creating a new name for an existing binding.
 \item \structure{Declaration:} Creating a completely new binding.
 \item in C: \texttt{struct Person} is a declaration. \texttt{typedef struct Person persontype}
 is a definition.
 \item in C++: \texttt{double x} is a declaration. \texttt{double \&y=x;} is a definition.
 \item creating a new entity or not. Usually the distinction is not clear and used
 interchangeably.
\end{itemize}
\end{frame}

\subsection{Sequential Declarations}
\begin{frame}
\frametitle{Sequential Declarations}
\begin{itemize}
 \item \texttt{D$_1$ ; D$_2$ ; ... ; D$_n$}
 \item Each declaration is available starting with the next line. \texttt{D$_1$} can be used in
 \texttt{D$_2$} an afterwards,  \texttt{D$_2$} can be used in \texttt{D$_3$} and afterwards,...
 \item Declared identifier is not available in preceding declarations.
 \item Most programming languages provide only such declarations.
\end{itemize}
\end{frame}

\subsection{Collateral Declarations}
\begin{frame}
\frametitle{Collateral Declarations}
\begin{itemize}
 \item \texttt{Start; D$_1$ and D$_2$ and ...  and T$_n$ ; End}
 \item Each declaration is evaluated in the environment preceding the declaration group.
 Declared identifiers are available only after all finish.
\texttt{D$_1$},... \texttt{D$_n$} uses in the environment of \texttt{Start}. They are 
available in the environment of \texttt{End}.
 \item ML allows collateral declarations additionally.
\end{itemize}
\end{frame}

\subsection{Recursive declarations}
\begin{frame}
\frametitle{Recursive declarations}
\begin{itemize}
 \item Declaration:\texttt{Name = Body}
 \item The body of the declaration can access the declared identifier. Declaration is available
 in the body of itself.
 \item C functions and type declarations are recursive. Variable definitions are usually not
 recursive. ML allows programmer to choose among recursive and non-recursive
 function definitions.
\end{itemize}
\end{frame}

\subsection{Recursive Collateral Declarations}
\begin{frame}
\frametitle{Recursive Collateral Declarations}
\begin{itemize}
 \item All declarations can access the others regardless of their order.
 \item All Haskell declarations are recursive collateral (including variables)
 \item All declarations are mutually recursive.
 \item ML allows programmer to do such definitions.
 \item C++ class members are like this.
 \item in C a similar functionality can be achieved by prototype definitions.
\end{itemize}
\end{frame}

\subsection{Block Expressions}
\begin{frame}[fragile]
\frametitle{Block Expressions}
\begin{itemize}
 \item Allows an expression to be evaluated in a special local environment. Declarations done
 in the block is not available outside.
 \item in Haskell: \texttt{let D$_1$; D$_2$;  ... ; D$_n$ in {\em Expression\/} } or
 \texttt{{\em Expression\/} where D$_1$; D$_2$;  ... ; D$_n$} 
\begin{beamercolorbox}{hexample}
\begin{lstlisting}[language={Haskell},escapechar=\#]
x=5
t=let xsquare=x*x
      factorial n = if n<2 then 1 else n*factorial (n-1)
      xfact = factorial x
  in  (xsquare+1)*xfact/(xfact*xsquare+2)
\end{lstlisting}\end{beamercolorbox}
\end{itemize}
\end{frame}

\begin{frame}[fragile]
\begin{itemize}
 \item Hiding works in block expressions as expected:
\begin{beamercolorbox}{hexample}
\begin{lstlisting}[language={Haskell},escapechar=\#]
x=5 ; y=6 ; z = 3
t=let x=1 
  in let y=2 
     in   x+y+z
{-- t is 1+2+3 here. local x and y hides the ones above --}
\end{lstlisting}
\end{beamercolorbox}
 \item GCC (only GCC) block expressions has the last expression in block as the value:
\begin{beamercolorbox}{cexample}
\begin{lstlisting}[language=C,escapechar=\#]
 double min ;
 ...
 min = ({ double tmp;
          if (b < a) then {
             tmp = a;  a = b ; b = tmp;
          }
          a; // this is the value of the block
         });
\end{lstlisting}\end{beamercolorbox}
\end{itemize}
\end{frame}

\defverbatim[colored]\codeblokcomC{
\begin{lstlisting}[language={C},escapechar=\#]
int x=3, i=2;
x += i;
while (x>i) { 
    int i=0;
    ...
    i++;
}
/* i is 2 again */
\end{lstlisting}}
\subsection{Block Commands}
\begin{frame}
\frametitle{Block Commands}
\begin{itemize}
 \item Similar to block expressions, declarations done inside a block command is available only
 during the block. Statements inside work in this environment. The declarations lost outside of
 the block. 
 \item \ \\
\begin{beamercolorbox}{cexample}
\codeblokcomC
\end{beamercolorbox}
\end{itemize}

\end{frame}

\defverbatim[colored]\codeblokdecH{
\begin{lstlisting}[language={Haskell},escapechar=\#]
fifthpower x = (forthpowerx) * x where
		squarex = x*x
		forthpowerx = squarex*squarex
\end{lstlisting}}
\subsection{Block Declarations}
\begin{frame}
\frametitle{Block Declarations}
\begin{itemize}
 \item A declaration is made in a local environment of declarations. Local declarations
 	are not made available to the outer environment.
 \item in Haskell: \texttt{D$_{exp}$ where D$_1$; D$_2$;  ... ; D$_n$}\\
 	Only \texttt{D$_{exp}$} is added to environment. Body of \texttt{D$_{exp}$} has all local
	declarations available in its environment.
\begin{beamercolorbox}{hexample}
 \codeblokdecH
\end{beamercolorbox}
\end{itemize}
\end{frame}

\section{Summary}
\begin{frame}
\frametitle{Summary}
\begin{itemize}
\item Binding, scope, environment
\item Block structure
\item Hiding
\item Static vs Dynamic binding
\item Declarations
\item Sequential, recursive, collateral
\item Expression, command and declaration blocks
\end{itemize}
\end{frame}
\end{document}
